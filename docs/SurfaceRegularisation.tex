% SurfaceRegularisation.tex
%
\documentclass[12pt,a4paper]{article}

 \begin{document}

  \title{Cortical Surface Mesh Regularisation.}
  \author{S.A. Knock}  % \and A.N. Researcher \and G.W. Payde
  \date{\today}
  \maketitle

   Regularisation:
   \begin{enumerate}
    \item Check for and, if necessary, correct topological anomalies.
    \item Split long edges, for a cortex of 150000+ vertices per 
          hemisphere, all edges longer than 2mm should be split.
    \item Shift vertices toward the centre of mass of their 1-ring, 
          repeat several times with weighting between 0.3 and 0.5.
    \item Recheck for long edges, split them.
    \item Shift vertices toward centre of mass.
    \item Remove degree 3 vertices where resultant triangle is smaller 
          than upper limit for triangle size.
    \item Collapse edges, targeting edges shorter than 0.1mm as well 
          as the shortest edges of high degree vertices.
    \item Shift vertices again.
    \item Final check for Long edges, short edges, degree 3, and 
          high degree vertices -- split, collapse and remove them 
          as necessary.
    \item Final shift of vertices. 
   \end{enumerate}
   This results in a regularised high resolution mesh, in order to 
   provide a set of progressively lower resolution meshes with 
   minimally different surface properties we need to simplify the 
   mesh with the same biased criteria as the regularisation process.
   However, it is important not to perform further shifting of the 
   vertices as, when performed on lower resolution meshes, this 
   procedure has an increasingly damaging effect on the mesh's 
   correspondence to the surface being represented.

  \end{document}
